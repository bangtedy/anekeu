%%%%%%%%%%%%%%%%%%%%%%%%%%%%%%%%%%%%%%%%%%%%%%%%%%%%%%%%%%%%%%%
%% OXFORD THESIS TEMPLATE

% Use this template to produce a standard thesis that meets the Oxford University requirements for DPhil submission
%
% Originally by Keith A. Gillow (gillow@maths.ox.ac.uk), 1997
% Modified by Sam Evans (sam@samuelevansresearch.org), 2007
% Modified by John McManigle (john@oxfordechoes.com), 2015
% Modified by Ulrik Lyngs (ulrik.lyngs@cs.ox.ac.uk), 2018-, for use with R Markdown
%
% Ulrik Lyngs, 25 Nov 2018: Following John McManigle, broad permissions are granted to use, modify, and distribute this software
% as specified in the MIT License included in this distribution's LICENSE file.
%
% John commented this file extensively, so read through to see how to use the various options.  Remember that in LaTeX,
% any line starting with a % is NOT executed.

%%%%% PAGE LAYOUT
% The most common choices should be below.  You can also do other things, like replace "a4paper" with "letterpaper", etc.

% 'twoside' formats for two-sided binding (ie left and right pages have mirror margins; blank pages inserted where needed):
%\documentclass[a4paper,twoside]{templates/ociamthesis}
% Specifying nothing formats for one-sided binding (ie left margin > right margin; no extra blank pages):
%\documentclass[a4paper]{ociamthesis}
% 'nobind' formats for PDF output (ie equal margins, no extra blank pages):
%\documentclass[a4paper,nobind]{templates/ociamthesis}

% As you can see from the line below, oxforddown uses the a4paper size, 
% and passes in the binding option from the YAML header in index.Rmd:
\documentclass[a4paper, nobind]{templates/ociamthesis}


%%%%% ADDING LATEX PACKAGES
% add hyperref package with options from YAML %
\usepackage[pdfpagelabels]{hyperref}
% handle long urls
\usepackage{xurl}
% change the default coloring of links to something sensible
\usepackage{xcolor}

\definecolor{mylinkcolor}{RGB}{0,0,139}
\definecolor{myurlcolor}{RGB}{0,0,139}
\definecolor{mycitecolor}{RGB}{0,33,71}

\hypersetup{
  hidelinks,
  colorlinks,
  linktocpage=true,
  linkcolor=mylinkcolor,
  urlcolor=myurlcolor,
  citecolor=mycitecolor
}


% add float package to allow manual control of figure positioning %
\usepackage{float}

% enable strikethrough
\usepackage[normalem]{ulem}

% use soul package for correction highlighting
\usepackage{color, soul}
\definecolor{correctioncolor}{HTML}{CCCCFF}
\sethlcolor{correctioncolor}
\newcommand{\ctext}[3][RGB]{%
  \begingroup
  \definecolor{hlcolor}{#1}{#2}\sethlcolor{hlcolor}%
  \hl{#3}%
  \endgroup
}
% stop soul from freaking out when it sees citation commands
\soulregister\ref7
\soulregister\cite7
\soulregister\citet7
\soulregister\autocite7
\soulregister\textcite7
\soulregister\pageref7

%%%%% FIXING / ADDING THINGS THAT'S SPECIAL TO R MARKDOWN'S USE OF LATEX TEMPLATES
% pandoc puts lists in 'tightlist' command when no space between bullet points in Rmd file,
% so we add this command to the template
\providecommand{\tightlist}{%
  \setlength{\itemsep}{0pt}\setlength{\parskip}{0pt}}
 
% allow us to include code blocks in shaded environments
\usepackage{color}
\usepackage{fancyvrb}
\newcommand{\VerbBar}{|}
\newcommand{\VERB}{\Verb[commandchars=\\\{\}]}
\DefineVerbatimEnvironment{Highlighting}{Verbatim}{commandchars=\\\{\}}
% Add ',fontsize=\small' for more characters per line
\usepackage{framed}
\definecolor{shadecolor}{RGB}{248,248,248}
\newenvironment{Shaded}{\begin{snugshade}}{\end{snugshade}}
\newcommand{\AlertTok}[1]{\textcolor[rgb]{0.94,0.16,0.16}{#1}}
\newcommand{\AnnotationTok}[1]{\textcolor[rgb]{0.56,0.35,0.01}{\textbf{\textit{#1}}}}
\newcommand{\AttributeTok}[1]{\textcolor[rgb]{0.13,0.29,0.53}{#1}}
\newcommand{\BaseNTok}[1]{\textcolor[rgb]{0.00,0.00,0.81}{#1}}
\newcommand{\BuiltInTok}[1]{#1}
\newcommand{\CharTok}[1]{\textcolor[rgb]{0.31,0.60,0.02}{#1}}
\newcommand{\CommentTok}[1]{\textcolor[rgb]{0.56,0.35,0.01}{\textit{#1}}}
\newcommand{\CommentVarTok}[1]{\textcolor[rgb]{0.56,0.35,0.01}{\textbf{\textit{#1}}}}
\newcommand{\ConstantTok}[1]{\textcolor[rgb]{0.56,0.35,0.01}{#1}}
\newcommand{\ControlFlowTok}[1]{\textcolor[rgb]{0.13,0.29,0.53}{\textbf{#1}}}
\newcommand{\DataTypeTok}[1]{\textcolor[rgb]{0.13,0.29,0.53}{#1}}
\newcommand{\DecValTok}[1]{\textcolor[rgb]{0.00,0.00,0.81}{#1}}
\newcommand{\DocumentationTok}[1]{\textcolor[rgb]{0.56,0.35,0.01}{\textbf{\textit{#1}}}}
\newcommand{\ErrorTok}[1]{\textcolor[rgb]{0.64,0.00,0.00}{\textbf{#1}}}
\newcommand{\ExtensionTok}[1]{#1}
\newcommand{\FloatTok}[1]{\textcolor[rgb]{0.00,0.00,0.81}{#1}}
\newcommand{\FunctionTok}[1]{\textcolor[rgb]{0.13,0.29,0.53}{\textbf{#1}}}
\newcommand{\ImportTok}[1]{#1}
\newcommand{\InformationTok}[1]{\textcolor[rgb]{0.56,0.35,0.01}{\textbf{\textit{#1}}}}
\newcommand{\KeywordTok}[1]{\textcolor[rgb]{0.13,0.29,0.53}{\textbf{#1}}}
\newcommand{\NormalTok}[1]{#1}
\newcommand{\OperatorTok}[1]{\textcolor[rgb]{0.81,0.36,0.00}{\textbf{#1}}}
\newcommand{\OtherTok}[1]{\textcolor[rgb]{0.56,0.35,0.01}{#1}}
\newcommand{\PreprocessorTok}[1]{\textcolor[rgb]{0.56,0.35,0.01}{\textit{#1}}}
\newcommand{\RegionMarkerTok}[1]{#1}
\newcommand{\SpecialCharTok}[1]{\textcolor[rgb]{0.81,0.36,0.00}{\textbf{#1}}}
\newcommand{\SpecialStringTok}[1]{\textcolor[rgb]{0.31,0.60,0.02}{#1}}
\newcommand{\StringTok}[1]{\textcolor[rgb]{0.31,0.60,0.02}{#1}}
\newcommand{\VariableTok}[1]{\textcolor[rgb]{0.00,0.00,0.00}{#1}}
\newcommand{\VerbatimStringTok}[1]{\textcolor[rgb]{0.31,0.60,0.02}{#1}}
\newcommand{\WarningTok}[1]{\textcolor[rgb]{0.56,0.35,0.01}{\textbf{\textit{#1}}}}

% set white space before and after code blocks


\renewenvironment{Shaded}
{
  \vspace{10pt}%
  \begin{snugshade}%
}{%
  \end{snugshade}%
  \vspace{8pt}%
}

% User-included things with header_includes or in_header will appear here
% kableExtra packages will appear here if you use library(kableExtra)
\usepackage{fontspec}
\setmainfont{Arial}
\usepackage{booktabs}
\usepackage{longtable}
\usepackage{array}
\usepackage{multirow}
\usepackage{wrapfig}
\usepackage{float}
\usepackage{colortbl}
\usepackage{pdflscape}
\usepackage{tabu}
\usepackage{threeparttable}
\usepackage{threeparttablex}
\usepackage[normalem]{ulem}
\usepackage{makecell}
\usepackage{xcolor}


%UL set section header spacing
\usepackage{titlesec}
% 
\titlespacing\subsubsection{0pt}{24pt plus 4pt minus 2pt}{0pt plus 2pt minus 2pt}


%UL set whitespace around verbatim environments
\usepackage{etoolbox}
\makeatletter
\preto{\@verbatim}{\topsep=0pt \partopsep=0pt }
\makeatother


%%%%%%% PAGE HEADERS AND FOOTERS %%%%%%%%%
\usepackage{fancyhdr}
\setlength{\headheight}{15pt}
\fancyhf{} % clear the header and footers
\pagestyle{fancy}
\renewcommand{\chaptermark}[1]{\markboth{\thechapter. #1}{\thechapter. #1}}
\renewcommand{\sectionmark}[1]{\markright{\thesection. #1}} 
\renewcommand{\headrulewidth}{0pt}

\fancyhead[LO]{\emph{\leftmark}} 
\fancyhead[RE]{\emph{\rightmark}} 




% UL page number position 
\fancyhead[R]{\emph{\thepage}} %regular pages
\fancypagestyle{plain}{\fancyhf{}\fancyfoot[L]{\emph{\thepage}}} %chapter pages




%%%%% SELECT YOUR DRAFT OPTIONS
% This adds a "DRAFT" footer to every normal page.  (The first page of each chapter is not a "normal" page.)

% IP feb 2021: option to include line numbers in PDF

% for line wrapping in code blocks
\usepackage{fancyvrb}
\usepackage{fvextra}
\DefineVerbatimEnvironment{Highlighting}{Verbatim}{breaklines=true, breakanywhere=true, commandchars=\\\{\}}

% for quotations -- loaded here rather than in ociamthesis.cls, as it needs to
% be loaded after fvextra, otherwise we get a warning message
\usepackage{csquotes}

% This highlights (in blue) corrections marked with (for words) \mccorrect{blah} or (for whole
% paragraphs) \begin{mccorrection} . . . \end{mccorrection}.  This can be useful for sending a PDF of
% your corrected thesis to your examiners for review.  Turn it off, and the blue disappears.
\correctionstrue


%%%%% BIBLIOGRAPHY SETUP
% Note that your bibliography will require some tweaking depending on your department, preferred format, etc.
% If you've not used LaTeX before, I recommend just using pandoc for citations -- this is what's used unless you specific e.g. "citation_package: natbib" in index.Rmd
% If you're already a LaTeX pro and are used to natbib or something, modify as necessary.

% this allows the latex template to handle pandoc citations
\newlength{\cslhangindent}
\setlength{\cslhangindent}{1.5em}
\newlength{\csllabelwidth}
\setlength{\csllabelwidth}{3em}
\newlength{\cslentryspacingunit} % times entry-spacing
\setlength{\cslentryspacingunit}{\parskip}
\newenvironment{CSLReferences}[2] % #1 hanging-ident, #2 entry spacing
 {% don't indent paragraphs
  \setlength{\parindent}{0pt}
  % turn on hanging indent if param 1 is 1
  \ifodd #1
  \let\oldpar\par
  \def\par{\hangindent=\cslhangindent\oldpar}
  \fi
  % set entry spacing
  \setlength{\parskip}{1mm}
  \setlength{\baselineskip}{6mm}
 }%
 {}
\usepackage{calc}
\newcommand{\CSLBlock}[1]{#1\hfill\break}
\newcommand{\CSLLeftMargin}[1]{\parbox[t]{\csllabelwidth}{#1}}
\newcommand{\CSLRightInline}[1]{\parbox[t]{\linewidth - \csllabelwidth}{#1}\break}
\newcommand{\CSLIndent}[1]{\hspace{\cslhangindent}#1}




% Uncomment this if you want equation numbers per section (2.3.12), instead of per chapter (2.18):
%\numberwithin{equation}{subsection}


%%%%% THESIS / TITLE PAGE INFORMATION
% Everybody needs to complete the following:
\title{\texttt{RStudio}:\\
untuk Ekononomi dan Bisnis\\}
\author{Tedy Herlambang}
\college{---}

% Master's candidates who require the alternate title page (with candidate number and word count)
% must also un-comment and complete the following three lines:

% Uncomment the following line if your degree also includes exams (eg most masters):
%\renewcommand{\submittedtext}{Submitted in partial completion of the}
% Your full degree name.  (But remember that DPhils aren't "in" anything.  They're just DPhils.)
\degree{---}

% Term and year of submission, or date if your board requires (eg most masters)
\degreedate{2025}


%%%%% YOUR OWN PERSONAL MACROS
% This is a good place to dump your own LaTeX macros as they come up.

% To make text superscripts shortcuts
\renewcommand{\th}{\textsuperscript{th}} % ex: I won 4\th place
\newcommand{\nd}{\textsuperscript{nd}}
\renewcommand{\st}{\textsuperscript{st}}
\newcommand{\rd}{\textsuperscript{rd}}

%%%%% THE ACTUAL DOCUMENT STARTS HERE
\begin{document}

%%%%% CHOOSE YOUR LINE SPACING HERE
% This is the official option.  Use it for your submission copy and library copy:
\setlength{\textbaselineskip}{22pt plus2pt}
% This is closer spacing (about 1.5-spaced) that you might prefer for your personal copies:
%\setlength{\textbaselineskip}{18pt plus2pt minus1pt}

% You can set the spacing here for the roman-numbered pages (acknowledgements, table of contents, etc.)
\setlength{\frontmatterbaselineskip}{17pt plus1pt minus1pt}

% UL: You can set the line and paragraph spacing here for the separate abstract page to be handed in to Examination schools
\setlength{\abstractseparatelineskip}{13pt plus1pt minus1pt}
\setlength{\abstractseparateparskip}{0pt plus 1pt}

% UL: You can set the general paragraph spacing here - I've set it to 2pt (was 0) so
% it's less claustrophobic
\setlength{\parskip}{2pt plus 1pt}

%
% Customise title page
%
\def\crest{}
\renewcommand{\university}{---}
\renewcommand{\submittedtext}{---}
\renewcommand{\thesistitlesize}{\fontsize{22pt}{28pt}\selectfont}
\renewcommand{\gapbeforecrest}{25mm}
\renewcommand{\gapaftercrest}{25mm
}


% Leave this line alone; it gets things started for the real document.
\setlength{\baselineskip}{\textbaselineskip}


%%%%% CHOOSE YOUR SECTION NUMBERING DEPTH HERE
% You have two choices.  First, how far down are sections numbered?  (Below that, they're named but
% don't get numbers.)  Second, what level of section appears in the table of contents?  These don't have
% to match: you can have numbered sections that don't show up in the ToC, or unnumbered sections that
% do.  Throughout, 0 = chapter; 1 = section; 2 = subsection; 3 = subsubsection, 4 = paragraph...

% The level that gets a number:
\setcounter{secnumdepth}{2}
% The level that shows up in the ToC:
\setcounter{tocdepth}{1}


%%%%% ABSTRACT SEPARATE
% This is used to create the separate, one-page abstract that you are required to hand into the Exam
% Schools.  You can comment it out to generate a PDF for printing or whatnot.

% JEM: Pages are roman numbered from here, though page numbers are invisible until ToC.  This is in
% keeping with most typesetting conventions.
\begin{romanpages}

% Title page is created here
\maketitle

%%%%% DEDICATION
\begin{dedication}
  For The Rising Star
\end{dedication}

%%%%% ACKNOWLEDGEMENTS


\begin{acknowledgements}
 	Terima kasih yang sebesar-besarnya saya sampaikan kepada: Prof.~Abdul Haris, Dr.~Titik Musriati dan Dr.~Ngatimun (Universitas Panca Marga), Muhammad Agus Nugroho dan Rifaldi Kadir (UIN Gorontalo), Rudi Masniadi (Universitas Teknologi Sumbawa), Erlyn Yuniashri dan Ajeng Kartika Galuh (Universitas Brawijaya Malang), Oktaviani Ika Wijayanti, Frederic Winston Nalle (Universitas Timor), Yeni Puspita (Universitas Negeri Jember) dan Muhammad Rizal (Universitas Islam Malang).

 \begin{flushright}
 Tedy Herlambang \\
 UPM \\
 4 Januari 2024
 \end{flushright}
\end{acknowledgements}



%%%%% ABSTRACT


\renewcommand{\abstracttitle}{Sinopsis}
\begin{abstract}
	Buku ini merupakan \emph{pengantar} cara melakukan analisis ekonomi dan bisnis secara kuantitatif dengan menggunakan R dan RStudio. Persoalan yang dibahas umumnya tingkat S1 atau S2.

Saya berharap pembaca dapat mengambil satu gagasan dari buku ini dalam melakukan investigasi ekonomi dan bisnis secara kuantitatif: data dan alat-alat analisis tidak sempurna. Namun, ketika kita memahami kekuatan dan kelemahan alat-alat ini, kita dapat menggunakannya untuk menemukan hal-hal menarik di dalam ekonomi dan bisnis.

Jangan ragu untuk menghubungi saya dengan pemikiran Anda tentang buku ini, ide perbaikan/materi tambahan atau mungkin kesalahan yang Anda temukan di buku ini.
\end{abstract}



%%%%% MINI TABLES
% This lays the groundwork for per-chapter, mini tables of contents.  Comment the following line
% (and remove \minitoc from the chapter files) if you don't want this.  Un-comment either of the
% next two lines if you want a per-chapter list of figures or tables.
\dominitoc % include a mini table of contents

% This aligns the bottom of the text of each page.  It generally makes things look better.
\flushbottom

% This is where the whole-document ToC appears:
\tableofcontents

\listoffigures
	\mtcaddchapter
  	% \mtcaddchapter is needed when adding a non-chapter (but chapter-like) entity to avoid confusing minitoc

% Uncomment to generate a list of tables:
\listoftables
  \mtcaddchapter
%%%%% LIST OF ABBREVIATIONS
% This example includes a list of abbreviations.  Look at text/abbreviations.tex to see how that file is
% formatted.  The template can handle any kind of list though, so this might be a good place for a
% glossary, etc.
% First parameter can be changed eg to "Glossary" or something.
% Second parameter is the max length of bold terms.
\begin{mclistof}{Daftar Istilah}{3.2cm}

\item[1-D, 2-D]

\begin{center}\rule{0.5\linewidth}{0.5pt}\end{center}

\item[Otter]

\begin{center}\rule{0.5\linewidth}{0.5pt}\end{center}

\item[Hedgehog]

\begin{center}\rule{0.5\linewidth}{0.5pt}\end{center}

\end{mclistof} 


% The Roman pages, like the Roman Empire, must come to its inevitable close.
\end{romanpages}

%%%%% CHAPTERS
% Add or remove any chapters you'd like here, by file name (excluding '.tex'):
\flushbottom

% all your chapters and appendices will appear here
\hypertarget{pengenalan-r-dan-rstudio}{%
\chapter{Pengenalan R dan Rstudio}\label{pengenalan-r-dan-rstudio}}

\minitoc 

Ada banyak situs rujukan bagus untuk belajar R. Tempat pertama menurut saya adalah \emph{R Project for Statistical Computing} di \url{http://www.r-project.org/}. Situs ini berisi tautan ke buku, manual, \emph{R Journal} dan lain-lain. Silakan unduh R di situs ini. Gratis! Sedangkan untuk menginstal RStudio silakan kunjungi \href{https://posit.co/download/rstudio-desktop/}{posit}. Silakan unduh \emph{RStudio Desktop} dari situs ini. Juga gratis! Untuk pengenalan ini, saya berasumsi bahwa Anda telah mengunduh dan menginstal R dan RStudio di komputer Anda. Kurva pembelajaran untuk R tidak terlalu curam. Dalam waktu singkat, Anda akan dapat menggunakan R secara aktif.

R tersedia secara gratis untuk berbagai platform komputasi seperti Windows, Macintosh dan Unix. Ini membuat R lebih fleksibel. Tetapi di sisi lain, ini berarti tidak ada pengembangan antarmuka pengguna grafis (GUI) yang sesuai dengan keinginan pengguna seperti halnya program berbayar SPSS atau Stata. GUI bawaan R terlihat lebih mirip dengan konsol DOS lama. Lihat Gambar \ref{fig:rgui}.

\begin{figure}[H]
\includegraphics[width=1\linewidth]{figures/console432} \caption{GUI R}\label{fig:rgui}
\end{figure}

Dalam buku ini, kita akan fokus menggunakan R melalui antarmuka \textbf{RStudio}. Kita bisa menulis skrip menggunakan teks editor selain RStudio. Tetapi dengan RStudio, R lebih mudah digunakan dan lebih mirip dengan program SPSS atau Stata.

RStudio bukan R, juga tidak menyertakan R saat Anda mengunduh dan menginstalnya. Oleh karena itu, untuk menggunakan RStudio kita perlu menginstal R terlebih dahulu. Hubungan antara R dan RStudio seperti metafora mesin mobil dan dashboard (Ismay \& Kim (\protect\hyperlink{ref-ismayStatisticalInferenceData2019}{2019})). Bahasa R dapat diumpamakan sebagai mesin dalam mobil yang bekerja, sedangkan RStudio sebagai dasbornya yang menyediakan cara mudah untuk mengonfigurasi dan mengontrol mesin.

Saya tidak akan menunjukkan pengenalan mendalam tentang fitur-fitur R atau RStudio. Apa yang akan saya bahas disini hanya menyentuh permukaan saja dari kemampuan R dan RStudio yang sangat banyak. Ada banyak rujukan bagus dan gratis untuk mempelajari dasar-dasarnya secara \emph{online}. Untuk pengantar R dapat ditemukan di Crawley (\protect\hyperlink{ref-crawleyBook2012}{2012}) dan Wickham \& Grolemund (\protect\hyperlink{ref-wickhamDataScience2017}{2017}), dan untuk pengenalan RStudio dapat dibaca Verzani (\protect\hyperlink{ref-verzaniGettingStartedRStudio2011}{2011}).

\hypertarget{tata-letak-rstudio}{%
\section{Tata Letak RStudio}\label{tata-letak-rstudio}}

Sebagai permulaan, kita akan mempelajari tata letak RStudio dan elemen inti yang akan kita gunakan saat bekerja (nanti Anda dapat mengubah tata letak dan tampilan RStudio sesuai selera). Antarmuka RStudio dibagi menjadi beberapa kuadran (lihat Gambar \ref{fig:3qu}).

Saat Anda memulai RStudio untuk pertama kalinya, Anda akan melihat tiga panel. Panel kiri menunjukkan konsol R. Kuadran ini juga dapat digunakan sebagai sumber masukan langsung untuk perintah. Kita belum membuka skrip, jadi kita tidak melihat panel skrip. Anda mempunyai dua pilihan disini: mengetik langsung di konsol (panel di sebelah kiri) atau membuat skrip.

Jika kita mengetikkan perintah langsung ke konsol, kita menggunakan R secara interaktif atau R dalam gaya ``tanya jawab'' (Dalgaard (\protect\hyperlink{ref-dalgaardIntroductoryStatistics2008}{2008})). Kelemahannya cara ini adalah kita lebih sulit untuk mengulangi atau menyimpan perintah-perintah kita. Semua fungsi akan dijalankan segera setelah kita menekan tombol \textbf{Enter}. Jika kita ingin menjalankan perintah cepat yang tidak perlu didokumentasikan dalam skrip, kita cukup menggunakan cara ini yaitu memasukkan perintah tersebut di konsol dan tekan Enter (Windows).

Di sebelah kanan, panel atas berisi lima tab: \emph{Environment, History, Connections, Build} dan \emph{Tutorial}. Di dalam Environment semua kumpulan data, variabel, model, objek, dan plot yang berbeda akan disimpan. Kuadran ini juga menunjukkan daftar berbagai objek yang ada dalam sesi R yang sedang berjalan. Jika kita menginput suatu dataset atau membuat variabel baru, semuanya akan muncul di kuadran ini.

Sedangkan panel kanan bawah menampilkan enam tab: \emph{File, Plots, Packages, Help, Viewer} dan \emph{Presentation}. Anda dapat mengklik setiap tab untuk menelusuri berbagai fitur di dalamnya. Tab \emph{Plots} adalah tempat mencetak atau mengekspor plot dan grafik. Tab \emph{Packages} menunjukkan berbagai paket yang terpasang di komputer Anda. Tab yang penting disini adalah \emph{Help}: kita bisa bantuan jika kita lupa dengan fungsi-fungsi tertentu.

\begin{figure}[H]
\includegraphics[width=1\linewidth]{figures/3qu} \caption{Antar Muka RStudio}\label{fig:3qu}
\end{figure}

\hypertarget{r-dalam-gaya-tanya-jawab}{%
\section{R dalam Gaya ``Tanya Jawab''}\label{r-dalam-gaya-tanya-jawab}}

Disini kita akan mempraktiikan R dalam gaya tanya jawab yaitu mengetik perintah langsung ke konsol. Kita menggunakan gaya ini ketika hanya penghitungan rumus tunggal yang diperlukan, seperti kalkulator. Ketika komputasi melibatkan beberapa langkah, kita akan beralih menggunakan skrip (dibagian selanjutnya).

Dalam buku ini, kita jarang mengetik langsung ke konsol, namun jika Anda merasa perlu mengulangi atau mengunjungi kembali perintah yang telah Anda gunakan di konsol, di panel kanan atas ada tab \emph{History} berisi catatan lengkap dari setiap perintah yang telah Anda gunakan sebelumnya.

Sebagai ilustrasi, misalnya Anda ingin menghitung 2 × 2, kita menulis setelah \emph{command prompt} \textbf{\textgreater{}} 2*2 dan tekan \textbf{Enter}. Selanjutnya kita dapat memberi nama dan menyimpan hasilnya (sebagai nama variabel), misalnya \emph{opat}. Maka kita tuliskan opat \textless- 2*2. Artinya komputer menghitung 2 × 2 dan menyimpan hasilnya dalam variabel bernama opat. Perhatikan operator penugasan \textbf{\textless-} sebagai ciri khas R, walaupun kita bisa juga menggunakan tanda =. Contoh perintah/kode ditampilkan dalam kotak abu-abu, sedangkan luarannya didahului dengan tanda pagar ganda \#\#. Dengan cara yang sama kita buat variabel onom \textless- 2*3.

\begin{Shaded}
\begin{Highlighting}[]
\DecValTok{2}\SpecialCharTok{*}\DecValTok{2}
\end{Highlighting}
\end{Shaded}

\begin{verbatim}
## [1] 4
\end{verbatim}

\begin{Shaded}
\begin{Highlighting}[]
\NormalTok{opat }\OtherTok{\textless{}{-}} \DecValTok{2}\SpecialCharTok{*}\DecValTok{2}
\NormalTok{opat }\CommentTok{\# memanggil variabel opat}
\end{Highlighting}
\end{Shaded}

\begin{verbatim}
## [1] 4
\end{verbatim}

\begin{Shaded}
\begin{Highlighting}[]
\NormalTok{onom }\OtherTok{=} \DecValTok{3}\SpecialCharTok{*}\DecValTok{2}
\NormalTok{onom}
\end{Highlighting}
\end{Shaded}

\begin{verbatim}
## [1] 6
\end{verbatim}

Hal ini juga berlaku untuk operasi aritmatika (+, -, /, \^{},\ldots), operator perbandingan (==, \textless=,\ldots), operator logika (\&, \textbar, !,\ldots) dan fungsi matematika dasar seperti sin, cos, exp, logaritma (log), akar kuadrat (sqrt).

Variabel paling dasar dalam R adalah vektor. Vektor R adalah barisan nilai yang bertipe sama. Misalnya, untuk membuat vektor lima dimensi beranggotakan angka 7, kita menuliskan \emph{tujuh5 \textless-rep(7,5)}. Maka tujuh5 adalah vektor dengan masing-masing komponen 7; untuk melihatnya kita ketik \texttt{tujuh5} dan tekan Enter. Fungsi \emph{rep} merupakan fungsi khusus dari R yang merupakan kependekan dari pengulangan (repeat). Argumennya ditulis sebagai \emph{rep(apa yang diulang,berapa kali)}.

\begin{Shaded}
\begin{Highlighting}[]
\NormalTok{tujuh5 }\OtherTok{\textless{}{-}}\FunctionTok{rep}\NormalTok{(}\DecValTok{7}\NormalTok{,}\DecValTok{5}\NormalTok{)}
\NormalTok{tujuh5}
\end{Highlighting}
\end{Shaded}

\begin{verbatim}
## [1] 7 7 7 7 7
\end{verbatim}

Jika a dan b bilangan bulat, perinah a:b membuat vektor bilangan bulat dari a ke b.

\begin{Shaded}
\begin{Highlighting}[]
\NormalTok{vektor17 }\OtherTok{\textless{}{-}} \DecValTok{1}\SpecialCharTok{:}\DecValTok{7}
\NormalTok{vektor17}
\end{Highlighting}
\end{Shaded}

\begin{verbatim}
## [1] 1 2 3 4 5 6 7
\end{verbatim}

Untuk lebih bisa mengontrol vektor yang dibuat, Anda gunakan perintah sekuen \emph{seq}.

\begin{Shaded}
\begin{Highlighting}[]
\NormalTok{sek305 }\OtherTok{\textless{}{-}} \FunctionTok{seq}\NormalTok{(}\DecValTok{1}\NormalTok{,}\DecValTok{30}\NormalTok{,}\DecValTok{5}\NormalTok{) }\CommentTok{\# membuat vektor bilangan dari 1{-}30 dengan beda 5}
\NormalTok{sek305}
\end{Highlighting}
\end{Shaded}

\begin{verbatim}
## [1]  1  6 11 16 21 26
\end{verbatim}

\begin{Shaded}
\begin{Highlighting}[]
\NormalTok{sek301 }\OtherTok{\textless{}{-}} \FunctionTok{seq}\NormalTok{(}\DecValTok{30}\NormalTok{,}\DecValTok{1}\NormalTok{,}\SpecialCharTok{{-}}\DecValTok{5}\NormalTok{) }\CommentTok{\# membuat vektor bilangan dari 30{-}1 dengan beda {-}5}
\NormalTok{sek301}
\end{Highlighting}
\end{Shaded}

\begin{verbatim}
## [1] 30 25 20 15 10  5
\end{verbatim}

Elemen vektor dapat diindeks dengan tanda kurung {[} {]}. Argumen braket dapat berupa bilangan bulat tunggal atau vektor.

\begin{Shaded}
\begin{Highlighting}[]
\NormalTok{sek305[}\DecValTok{4}\NormalTok{]}
\end{Highlighting}
\end{Shaded}

\begin{verbatim}
## [1] 16
\end{verbatim}

\begin{Shaded}
\begin{Highlighting}[]
\NormalTok{sek305[}\DecValTok{1}\SpecialCharTok{:}\DecValTok{4}\NormalTok{]}
\end{Highlighting}
\end{Shaded}

\begin{verbatim}
## [1]  1  6 11 16
\end{verbatim}

\begin{Shaded}
\begin{Highlighting}[]
\NormalTok{sek305[}\FunctionTok{c}\NormalTok{(}\DecValTok{1}\NormalTok{,}\DecValTok{4}\NormalTok{,}\DecValTok{5}\NormalTok{,}\DecValTok{10}\NormalTok{)]}
\end{Highlighting}
\end{Shaded}

\begin{verbatim}
## [1]  1 16 21 NA
\end{verbatim}

Untuk perintah terakhir saya meminta elemen vektor sek305 yang pertama, keempat, kelima, dan kesepuluh. Karena tidak ada elemen kesepuluh di dalam vektor sek305, maka hasilnya NA (\emph{not available}). Seringkali kita juga ingin mencari elemen suatu vektor yang memenuhi suatu kondisi tertentu, misalnya elemen sek305 yang kurang dari 10. Hal ini dilakukan dengan meminta indeks yang memenuhi

\begin{Shaded}
\begin{Highlighting}[]
\NormalTok{sek305}
\end{Highlighting}
\end{Shaded}

\begin{verbatim}
## [1]  1  6 11 16 21 26
\end{verbatim}

\begin{Shaded}
\begin{Highlighting}[]
\FunctionTok{which}\NormalTok{(sek305 }\SpecialCharTok{\textless{}} \DecValTok{10}\NormalTok{) }\CommentTok{\# elemen keberapa yang kurang dari 10}
\end{Highlighting}
\end{Shaded}

\begin{verbatim}
## [1] 1 2
\end{verbatim}

\begin{Shaded}
\begin{Highlighting}[]
\NormalTok{index }\OtherTok{\textless{}{-}} \FunctionTok{which}\NormalTok{(sek305 }\SpecialCharTok{\textless{}} \DecValTok{10}\NormalTok{) }\CommentTok{\# bilangan berapa saja yang kurang dari 10 dari vektor sek305 }
\NormalTok{sek305[index]}
\end{Highlighting}
\end{Shaded}

\begin{verbatim}
## [1] 1 6
\end{verbatim}

Vektor dapat terdiri dari angka, karakter dan bahkan rangkaian karakter.

\begin{Shaded}
\begin{Highlighting}[]
\NormalTok{omahku }\OtherTok{\textless{}{-}} \FunctionTok{c}\NormalTok{(}\StringTok{"Malang"}\NormalTok{, }\StringTok{"memang"}\NormalTok{, }\StringTok{"kota"}\NormalTok{, }\StringTok{"yang"}\NormalTok{, }\StringTok{"sueejuk"}\NormalTok{) }\CommentTok{\# perhatikan penggunanaan ""}
\NormalTok{omahku}
\end{Highlighting}
\end{Shaded}

\begin{verbatim}
## [1] "Malang"  "memang"  "kota"    "yang"    "sueejuk"
\end{verbatim}

\begin{Shaded}
\begin{Highlighting}[]
\NormalTok{omahku[}\FunctionTok{c}\NormalTok{(}\DecValTok{3}\NormalTok{,}\DecValTok{4}\NormalTok{,}\DecValTok{5}\NormalTok{)]}
\end{Highlighting}
\end{Shaded}

\begin{verbatim}
## [1] "kota"    "yang"    "sueejuk"
\end{verbatim}

Untuk membuat matriks juga dapat dilakukan dengan mudah. Misalnya, \emph{mat.pi.10.5 \textless-matrix(pi, nrow=10,ncol=5)} akan membuat matriks 10 × 5 dengan nama \emph{mat.pi.10.5} dengan semua entri \(\pi\). Sekali lagi, untuk melihat hasilnya, ketik \emph{mat.pi.10.5} dan tekan Enter. R peka penulisan huruf besar-kecil. Jadi mat.pi.10.5 dan mat.Pi.10.5 memiliki arti yang berbeda. Semua perintah menggunakan huruf kecil, kecuali perintah \emph{View()}.

\begin{Shaded}
\begin{Highlighting}[]
\NormalTok{mat.pi.}\FloatTok{10.5} \OtherTok{\textless{}{-}}\FunctionTok{matrix}\NormalTok{(pi, }\AttributeTok{nrow=}\DecValTok{10}\NormalTok{,}\AttributeTok{ncol=}\DecValTok{5}\NormalTok{)}
\NormalTok{mat.pi.}\FloatTok{10.5}
\end{Highlighting}
\end{Shaded}

\begin{verbatim}
##           [,1]     [,2]     [,3]     [,4]     [,5]
##  [1,] 3.141593 3.141593 3.141593 3.141593 3.141593
##  [2,] 3.141593 3.141593 3.141593 3.141593 3.141593
##  [3,] 3.141593 3.141593 3.141593 3.141593 3.141593
##  [4,] 3.141593 3.141593 3.141593 3.141593 3.141593
##  [5,] 3.141593 3.141593 3.141593 3.141593 3.141593
##  [6,] 3.141593 3.141593 3.141593 3.141593 3.141593
##  [7,] 3.141593 3.141593 3.141593 3.141593 3.141593
##  [8,] 3.141593 3.141593 3.141593 3.141593 3.141593
##  [9,] 3.141593 3.141593 3.141593 3.141593 3.141593
## [10,] 3.141593 3.141593 3.141593 3.141593 3.141593
\end{verbatim}

\hypertarget{saatnya-menggunakan-skrip}{%
\section{Saatnya menggunakan skrip}\label{saatnya-menggunakan-skrip}}

Biasanya penghitungan statistik melibatkan banyak baris kode. Jika demikian kita beralih menggunakan skrip. Untuk memulai skrip baru, Anda dapat mengklik File, New File, lalu R Script atau \textbf{Ctrl+Shift+N}. Maka dilayar RStudio akan muncul layar baru di kiri atas sehingga Anda sekarang punya empat kuadran. Inilah salah satu keunggulan R dibandingkan perangkat lunak analisis \emph{point and click} yaitu Anda dapat menyimpan pekerjaan Anda sebagai skrip. Lihat Gambar \ref{fig:4qu}

\begin{figure}[H]
\includegraphics[width=1\linewidth]{figures/4qu} \caption{Antar Muka RStudio dengan Skrip}\label{fig:4qu}
\end{figure}

Skrip ini setara dengan file sintaks di Stata: skrip menyajikan kode yang diperlukan untuk menghasilkan analisis yang diperlukan oleh pengguna. Anda dapat menyimpan dan membagikan pekerjaan Anda sebagai skrip yang berisi kode yang dapat dijalankan lain waktu. Hal ini sangat berguna sebagai catatan analisis yang Anda lakukan, melakukan verifikasi analisis dan replikasi studi.

Untuk menjalankan kode dalam Skrip R, untuk satu baris kode, letakkan penunjuk tetikus di depan kode, untuk satu blok baris, pilih kode tersebut, lalu klik tombol \emph{Run} atau tekan Ctrl + Enter di Windows sistem.

Perintah dalam Skrip R dapat dengan mudah ditelusuri kembali, dimodifikasi, dan dibagikan dengan rekan kerja. Dalam Skrip R, dimungkinkan untuk menambahkan komentar menggunakan. Segala sesuatu yang mengikuti \# akan dianggap sebagai komentar dan tidak akan dijalankan oleh R.

Di awal setiap skrip R, sebaiknya diketikkan paket-paket yang diperlukan untuk mengimplementasikan kode dalam file. Setelah menulis kode untuk memuat paket dengan fungsi \emph{library()}, Anda dapat menambahkan, sebagai komentar, kata kunci untuk mengingatkan tentang penggunaan paket. Ini akan membantu kita mengingat isi file dan menjelaskan kepada orang lain apa yang diperlukan untuk mengimplementasikan kode dalam skrip R.

Ini juga merupakan kebiasan baik untuk mendeskripsikan proyek dan menulis komentar singkat di badan fungsi yang kita buat. Sekali lagi ini berguna bagi penulis skrip dan bagi orang ketiga yang akan membaca kode tersebut.

\hypertarget{manajemen-data-dan-projek-di-r}{%
\section{Manajemen Data dan Projek di R}\label{manajemen-data-dan-projek-di-r}}

R adalah bahasa pemrograman berorientasi objek. Saat Anda membukanya, Anda memiliki lingkungan (\emph{Environment}) kosong yang dapat diisi dengan objek sebanyak kemampuan memori komputer Anda. Segala sesuatu yang ingin Anda simpan atau manipulasi di lain waktu perlu didefinisikan sebagai objek di lingkungan ini. Termasuk disini file data, objek atau hasil model, grafik, dan sebagainya. Artinya, tidak seperti perangkat lunak statistik standar, yang biasanya hanya mengizinkan analis untuk membuka satu kumpulan data, R memungkinkan analis untuk bekerja dengan beberapa file data secara bersamaan.

Kita bisa mendapatkan direktori kerja saat ini menggunakan fungsi getwd(). Direktori kerja adalah cara kita memberi tahu R di mana mendapatkan file, dan juga di mana mengekspor file. Dengan kata lain, di sinilah R akan beroperasi di dalam komputer kita. Kita dapat mengatur direktori kerja kita dengan menentukan path file ke lokasi yang kita inginkan dengan perintah setwd, atau Anda dapat mengatur jalur menggunakan menu drop-down (Session -\textgreater{} Set Working Directory-\textgreater{} Choose Directory).

\begin{Shaded}
\begin{Highlighting}[]
\DocumentationTok{\#\# Working directory di atur ke desktop}
\FunctionTok{setwd}\NormalTok{(“}\SpecialCharTok{/}\NormalTok{Users}\SpecialCharTok{/}\NormalTok{Username}\SpecialCharTok{/}\NormalTok{Desktop”) }
\DocumentationTok{\#\# untuk memastikan working directory yang dipakai}
\FunctionTok{getwd}\NormalTok{() }
\NormalTok{[}\DecValTok{1}\NormalTok{] “}\SpecialCharTok{/}\NormalTok{Users}\SpecialCharTok{/}\NormalTok{Username}\SpecialCharTok{/}\NormalTok{Desktop” }
\end{Highlighting}
\end{Shaded}

Sering kali kita memiliki banyak file data sekaligus yang dimuat di lingkungan kita. Jika demikian manajemen data menjadi penting dalam konteks ini. Saya sangat menyarankan Anda membuat \emph{project} setiap kali Anda memulai pekerjaan yang berbeda, yang tidak terkait dengan pekerjaan sebelumnya. Projek disini adalah tempat untuk menyimpan pekerjaan Anda yang berkaitan dengan projek spesifik.

Untuk membuat proyek klik simbol R di pojok kanan atas, klik \emph{New Directory \textgreater{} New Project} lalu tulis nama direktori (misalnya rstudio4eb) dan klik \emph{Create project}. Ketika Anda sudah berada didalam projek folder, sebaiknya dibuat subfolder tambahan misalnya khusus untuk data, gafik, referensi dan lain-lain. Setelah folder ditentukan, aturlah R ke direktori proyek Anda, sehingga folder itu menjadi lokasi default tempat R mencari file.

Ekstensi file R untuk file data adalah .RData. Anda dapat menyimpan satu atau lebih objek dalam file tersebut menggunakan fungsi simpan (Ctrl + s). Di dalam \emph{RData} tersimpan hasil model, grafik, atau objek apa pun. Hal ini sangat berguna ketika model membutuhkan waktu lama untuk dalam penghitungannya. Hasil tersebut dapat disimpan dan digunakan kembali nanti.

Selain file RData, menyimpan file dalam format .csv juga efektif saat bekerja di R. Ada perintah dasar R untuk mengimpor dan mengekspor file csv (masing-masing \emph{read.csv} dan \emph{write.csv}). R dapat langsung mengimpor data dari paket perangkat lunak statistik lain (SPSS, Stata), namun hal ini memerlukan pengunduhan paket (\emph{package}) yang tidak disertakan dalam instalasi bawaan R. Kita membahas paket di bagian selanjutnya.

\hypertarget{paket-paket-r}{%
\section{Paket-Paket R}\label{paket-paket-r}}

R bekerja melalui paket termasuk paket dasar dan ribuan paket atau ekstensi. Dengan paket dasar dan ekstensinya, banyak analisis data statistik dapat dilakukan dan grafik statistik kualitas tinggi dihasilkan.

Karena sifatnya yang \emph{open source}, pengguna dapat mengembangkan dan memperluas fitur-fitur dasar R dengan penambahan fungsionalitas. Ketika seseorang ingin menambahkan fungsi tambahan ke R, mereka membuat sebuah paket: fitur tambahan yang dapat diinstal pada mesin pengguna dan kemudian dimuat ke R bila diperlukan. Paket juga dapat menyimpan fungsi dan kumpulan data R.

Anda perlu mengunduh paket tertentu untuk menyediakan perintah yang diperlukan dan dapat menjalankan model atau analisis tertentu jika fungsi itu tidak tersedia secara \emph{default}. Paket-paket itu memerlukan tiga langkah untuk menggunakannya. Pertama, kita harus menginstal paket itu sendiri, memuatnya dari \emph{library}, dan terakhir memanggil salah satu fungsi paket tersebut.

Paket dapat diunduh di CRAN (\emph{Comprehensive R Archive Network}) dan dapat diinstal melalui kode \emph{install.packages(``nama paket'')} atau melalui opsi drop-down (\emph{Tools -\textgreater{} Install Package}). Paket di CRAN telah dievaluasi untuk memastikan paket tersebut berfungsi di seluruh platform. Paket bisa juga tersedia bagi pengguna tetapi belum diarsipkan di CRAN tetapi masih berada di Github. Paket tersebut dapat diinstal menggunakan fungsi devtools::install\_ github (Wickham et al. (\protect\hyperlink{ref-wickhamDevtoolsToolsMake2022}{2022})). Untuk setiap sesi R yang baru, paket yang diperlukan perlu dimuat (bukan diinstal ulang, cukup dimuat ulang dengan fungsi \emph{library ()} atau \emph{require()}.

\hypertarget{mengimpor-data-ke-dalam-r}{%
\section{Mengimpor Data ke dalam R}\label{mengimpor-data-ke-dalam-r}}

Dalam pelaksanaan analisis data, data sering kali disimpan dalam berbagai format. R dapat membaca berbagai jenis file data seperti free format text files, comma separated value files (csv), file Excel, file SPSS, file SAS, dan file Stata. Mengimpornya ke dalam R juga mudah. Kita dapat mengimpor .csv file menggunakan basis R. Tetapi jika data dalam format lain, kita perlu menginstal paket untuk mengimpor data misalnya paket \emph{foreign} (Team et al. (\protect\hyperlink{ref-rcoreteamForeignReadData2023}{2023})) yang berfungsi untuk membaca dan menulis file data dari SAS, SPSS dan Stata. Saat memasukkan nama file, jangan lupa untuk menempatkan nama file diantara tanda kutip (``nama-file.csv'').

\startappendices

\hypertarget{beberapa-rujukan-untuk-belajar-r}{%
\chapter{Beberapa Rujukan untuk Belajar R}\label{beberapa-rujukan-untuk-belajar-r}}

\hypertarget{referensi}{%
\chapter*{Referensi}\label{referensi}}
\addcontentsline{toc}{chapter}{Referensi}

\markboth{Referensi}{}

\hypertarget{refs}{}
\begin{CSLReferences}{1}{0}
\leavevmode\vadjust pre{\hypertarget{ref-crawleyBook2012}{}}%
Crawley, M. J. (2012). \emph{The {R Book}} (2nd edition). {Wiley}.

\leavevmode\vadjust pre{\hypertarget{ref-dalgaardIntroductoryStatistics2008}{}}%
Dalgaard, P. (2008). \emph{Introductory {Statistics} with {R}}. {Springer}. \url{https://doi.org/10.1007/978-0-387-79054-1}

\leavevmode\vadjust pre{\hypertarget{ref-ismayStatisticalInferenceData2019}{}}%
Ismay, C., \& Kim, A. Y. (2019). \emph{Statistical {Inference} via {Data Science}: {A ModernDive} into {R} and the {Tidyverse}}. {CRC Press}.

\leavevmode\vadjust pre{\hypertarget{ref-rcoreteamForeignReadData2023}{}}%
Team, R. C., Bivand, R., Carey, V. J., DebRoy, S., Eglen, S., Guha, R., Herbrandt, S., Lewin-Koh, N., Myatt, M., Nelson, M., Pfaff, B., Quistorff, B., Warmerdam, F., Weigand, S., Foundation, F. S., \& Inc. (2023). \emph{Foreign: {Read Data Stored} by '{Minitab}', '{S}', '{SAS}', '{SPSS}', '{Stata}', '{Systat}', '{Weka}', '{dBase}', ...}

\leavevmode\vadjust pre{\hypertarget{ref-verzaniGettingStartedRStudio2011}{}}%
Verzani, J. (2011). \emph{Getting {Started} with {RStudio}}. {O'Reilly}.

\leavevmode\vadjust pre{\hypertarget{ref-wickhamDataScience2017}{}}%
Wickham, H., \& Grolemund, G. (2017). \emph{R for {Data Science}}. {O'Reilly Media}.

\leavevmode\vadjust pre{\hypertarget{ref-wickhamDevtoolsToolsMake2022}{}}%
Wickham, H., Hester, J., Chang, W., \& Byan, J. (2022). \emph{Devtools: {Tools} to {Make Developing R Packages Easier}}. CRAN.

\end{CSLReferences}

%%%%% REFERENCES


\end{document}
